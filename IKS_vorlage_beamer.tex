%% Encoding: ISO8859-1 %%

%% LaTeX-Beamer template for KIT design
%% by Erik Burger, Christian Hammer
%%
%% modified by Christian Henrich for IKS
%%
%% version 1.2
%%
%% mostly compatible to KIT corporate design v1.2
%% http://www.uni-karlsruhe.de/download/uka/Gestaltungsrichtlinien_komplett.pdf
%%
%% Problems, bugs and comments to
%% burger@ipd.uka.de

\documentclass[18pt]{beamer}
\usetheme{kit}

% remove the following line for German date style and logos

% if a custom picture is to be used on the title page, copy it into the 'logos'
% directory, in the line below, replace 'mypicture' with the 
% filename (without extension) and uncomment the line

% \renewcommand{\titleimage}{mypicture}
\usepackage{color}
\usepackage{tikz}
\usetikzlibrary{arrows.meta}
\usepackage{marvosym}
\usepackage{multimedia}
\def\checkmark{\tikz\fill[green][scale=0.4](0,.35) -- (.25,0) -- (1,.7) -- (.25,.15) -- cycle;}
\usepackage{pifont}% http://ctan.org/pkg/pifont
\newcommand{\xmark}{\color{red}{\ding{55}}}%
\DeclareMathOperator*{\argmax}{arg\,max}
\newcommand{\cfbox}[2]{%
    \colorlet{currentcolor}{.}%
    {\color{#1}%
    \fbox{\color{currentcolor}#2}}%
}
% (picture proportions: 63 : 20, *.eps format if you use latex+dvips+ps2pdf,
% *.jpg/*.png/*.pdf if you use pdflatex)

% if you want to see BibTeX keys in the references view instead of the symbol,
% uncomment the following line
% \usebibitemtemplate{\insertbiblabel}

% the presentation starts here

\title[Online Neural Network-based Language Identification]{Online Neural Network-based Language Identification}
\subtitle{Master's Thesis}
\author{Daniel H. Draper}

\institute[ISL]{Interactive Systems Lab - Institute for Anthropomatics and Robotics}
\setcounter{tocdepth}{1}
\begin{document}

\selectlanguage{english} % in der Regel einfach auskommentiert da stehen lassen
%\selectlanguage{ngerman} % in der Regel einfach auskommentiert da stehen lassen
%title page
\begin{frame}
\titlepage
\end{frame}

%table of contents
\frame{
\frametitle{Contents}
\tableofcontents
}

%% Encoding: ISO8859-1 %%

\section{Motivation}
\subsection{General}
\frame{
\frametitle{Motivation}
\begin{itemize}
\item ASR and SLT commercially great success [''OK, Google``, Siri, Jibbigo]
\pause
\item Best performing ASR models are still trained on one source language.\cite{tang2016multi}
\end{itemize}
\hrulefill
\pause
\begin{itemize}
\item Performance increase for multilingual tasks\cite{tang2016multi}, UI streamlining.
\end{itemize}
}
\subsection{Applications}
\frame{
\frametitle{Specific Applications}
\begin{block}{\includegraphics[width=0.85\textwidth,height=0.075\textwidth]{bilder/lectureTranslator}}
\begin{center}
\begin{itemize}
\item Low latency, online application
\pause
\item Small source language amount make performance increase likely
\end{itemize}
\end{center}
\end{block}
\pause
\begin{block}{\includegraphics[width=0.2\textwidth,height=0.08\textwidth]{bilder/EUParl}\huge{European Parliament}}
\begin{center}
\begin{itemize}
\item SLT already employed consistently\cite{tcstar}
\pause
\item Many languages challenging, great potential.
\end{itemize}
\end{center}
\end{block}
}

\section{Related Work}
\frame{
\frametitle{Related Work I}
\begin{block}{Historically}
\begin{center}
\begin{itemize}
\item PRLM\footnote{single-language phone recognition followed by language-dependent,
interpolated n-gram language modeling}, PPRM\footnote{multiple single-language phone recognizers and language-dependent parallel phone recognition}\cite{zissman1996comparison}
\pause
\item Gaussian Mixture Models (GMM)s \cite{torres2002approaches}
\end{itemize}
\end{center}
\end{block}
}
\frame{
\frametitle{Related Work II}
\begin{block}{Modern Approaches}
\begin{center}
\begin{itemize}
\item VSM\footnote{Vector Space Modelling} + PPRM\cite{4032773}
\pause
\item I-Vector approaches\cite{6680440}\cite{d2012phonotactic}
\end{itemize}
\end{center}
\end{block}
}
\frame{
\frametitle{Related Work III/Based on}
\begin{block}{Similar Approaches}
\begin{center}
\begin{itemize}
\item Matejka et. al\cite{matejka2014neural} similar setup: BNF \(\rightarrow\) LID, with averaging 5 LID nets. 
\pause
\item \cite{6289007} M. Heck et al. evaluate LID approaches prospect of Lecture Translator integration: PPRM/PRLM + Hybrid.
\end{itemize}
\end{center}
\end{block}
\hrulefill
\pause
\begin{itemize}
\item Experimental baseline setup from Markus \cite{Mueller2016b}: ''Language Adaptive DNNs for Improved Low Resource Speech Recognition``
\end{itemize}
}

\section{Theory of LID}
\frame{
\frametitle{Theory of LID}
\begin{itemize}
\item Given input \(S = \{ s(1), s(2), \dots , s(T)\}\), \(\{L_1,L_2,\dots,L_N\}\)
\end{itemize}
\pause
\begin{exampleblock}{Formulation of LID Task\cite{6451097}}
\(L^O = \argmax_I p(S|L_I)\)\newline
With segmentation into phones \(\upsilon\) (sequence \(\Upsilon\)):\newline
\pause
\(L^O = \argmax_I P(\Upsilon|L_I)\)\newline
With Viterbi decoding on set of phone models \(M\):\newline
\pause
\(\Upsilon^O = \argmax_{\upsilon} P(S|\upsilon, M)\)\newline
Combining last 2 equations:\newline
\pause
\cfbox{red}{
\begin{minipage}{4 in}
\(L^O = \argmax_{I}\sum_{\forall \Upsilon}P(S|\Upsilon, M)P(\Upsilon|L_I)\)
\end{minipage}}
\end{exampleblock}
}

\section{Experimental Setup}
\subsection{Overview}
\frame{
\frametitle{Experimental Setup}
\tikzstyle{layer}=[draw=black,fill=black!30]
\tikzstyle{layerlid}=[draw=black,fill=green!30]
\tikzstyle{dots}=[draw=black,fill=black]

   \begin{tikzpicture}[scale=0.8]

   % Acoustic feature stack

   \draw (-1.75, 2.5) node[draw=white,fill=white] {AF stack};
   \fill[layer] (-2,1) coordinate(l0bl) -- (-1.5,1) coordinate(l0br) --
(-1.5,2) coordinate(l0tr) -- (-2,2) coordinate(l0tl) -- (-2,1);

   \fill[layer] (-2,0) coordinate(l0_1bl) -- (-1.5,0) coordinate(l0_1br)
-- (-1.5,1) coordinate(l0_1tr) -- (-2,1) coordinate(l0_1tl) -- (-2,0);

   \draw[dots] (-1.75,-0.2) circle (0.045);
   \draw[dots] (-1.75,-0.375) circle (0.045);
   \draw[dots] (-1.75,-0.55) circle (0.045);

   \fill[layer] (-2,-1.75) coordinate(l0_2bl) -- (-1.5,-1.75)
coordinate(l0_2br) -- (-1.5,-0.75) coordinate(l0_2tr) -- (-2,-0.75)
coordinate(l0_2tl) -- (-2,-1.75);

   % Bottle-neck stack

   \fill[layer] (4,-1.5) coordinate(l5_1bl) -- (4.5,-1.5)
coordinate(l5_1br) -- (4.5,-0.5) coordinate(l5_1tr) -- (4,-0.5)
coordinate(l5_1tl) -- (4,-1.5);

   \draw[dots] (4.25,-1.7) circle (0.045);
   \draw[dots] (4.25,-1.875) circle (0.045);
   \draw[dots] (4.25,-2.05) circle (0.045);

   \fill[layer] (4,-3.25) coordinate(l5_2bl) -- (4.5,-3.25)
coordinate(l5_2br) -- (4.5,-2.25) coordinate(l5_2tr) -- (4,-2.25)
coordinate(l5_2tl) -- (4,-3.25);

   \draw (1.5,-2.25) node {DBNF};

   \draw (4.25,2.5) node[draw=white,fill=white] {BNF stack};

   % BNF Layers

   \fill[layer] (4,-0.5) coordinate(l5bl) -- (4.5,-0.5) coordinate(l5br)
-- (4.5,0.5) coordinate(l5tr) -- (4,0.5) coordinate(l5tl) -- (4,-0.5);
   \fill[layer] (2.5,-1.5) coordinate(l4bl) -- (3,-1.5) coordinate(l4br)
-- (3,1.5) coordinate(l4tr) -- (2.5,1.5) coordinate(l4tl) -- (2.5,-1.5);

   \draw[dots] (2,0) circle (0.045);
   \draw[dots] (2.175,0) circle (0.045);
   \draw[dots] (1.825,0) circle (0.045);

   \fill[layer] (1,-1.5) coordinate(l2bl) -- (1.5,-1.5) coordinate(l2br)
-- (1.5,1.5) coordinate(l2tr) -- (1,1.5) coordinate(l2tl) -- (1,-1.5);
   \fill[layer] (0,-1.5) coordinate(l1bl) -- (0.5,-1.5) coordinate(l1br)
-- (0.5,1.5) coordinate(l1tr) -- (0,1.5) coordinate(l1tl) -- (0,-1.5);

   \draw (l0tr) -- (l1bl);
   \draw (l0_2br) -- (l1tl);
   \draw (l1tr) -- (l2bl);
   \draw (l1br) -- (l2tl);
   \draw (l4tr) -- (l5bl);
   \draw (l4br) -- (l5tl);


   % DNN Layers

   \fill[layer] (6,-3) coordinate(l6bl) -- (6.5,-3) coordinate(l6br) --
(6.5,0) coordinate(l6tr) -- (6,0) coordinate(l6tl) -- (6,-3);
   \fill[layer] (7,-3) coordinate(l7bl) -- (7.5,-3) coordinate(l7br) --
(7.5,0) coordinate(l7tr) -- (7,0) coordinate(l7tl) -- (7,-3);

   \draw[dots] (8,-1.5) circle (0.045);
   \draw[dots] (8.175,-1.5) circle (0.045);
   \draw[dots] (7.825,-1.5) circle (0.045);

   \fill[layer] (8.5,-3) coordinate(l11bl) -- (9,-3) coordinate(l11br)
-- (9,0) coordinate(l11tr) -- (8.5,0) coordinate(l11tl) -- (8.5,-3);
   \fill[layer] (10.5,-2) coordinate(l12bl) -- (11,-2) coordinate(l12br)
-- (11,-1) coordinate(l12tr) -- (10.5,-1) coordinate(l12tl) -- (10.5,-1);

   \draw (10.75,2.5) node[draw=white,fill=white] {LID};

   \draw (l5tr) -- (l6bl);
   \draw (l5_2br) -- (l6tl);

   \draw (l6tr) -- (l7bl);
   \draw (l6br) -- (l7tl);
   \draw (l11tr) -- (l12bl);
   \draw (l11br) -- (l12tl);

   \draw (l12tl) -- (l12bl);

   \draw (7.5,-3.75) node {Language Identity Network};

   \end{tikzpicture}
}
\subsection{Data Sets}
\frame{
\frametitle{Data Sets}
\begin{block}{Euronews 2014}
\begin{center}
\begin{itemize}
\item 10 Languages (EN, FR, DE, AR, ES, IT, PO, PT, RU, TR)
\item Original Corpus: 72h / Language, Reduced corpus: 18h / Language (Random Sampling of 10.000 speakers)
\item 80 \% train data, 10 \% each dev/test set.
\end{itemize}
\end{center}
\end{block}
\pause
\begin{block}{Lecture Data}
\begin{center}
\begin{itemize}
\item 3 Languages (EN, FR, DE) ~10h per language.
\item KIT lectures, InterACT25, DGA talks
\item Evaluation of EUNews results, Proof-of-concept for Lecture Translator integration.
\end{itemize}
\end{center}
\end{block}
}
\frame{
\frametitle{Data Sets II}
\begin{block}{European Parliament}
\begin{center}
\begin{itemize}
\item 7 Languages (EN, FR, DE, ES, IT, PO, PT) 3.6h per language
\item Simultaneous translation of all EUParliament speeches
\item Further Evaluation, Proof-of-concept for EUParl integration.
\end{itemize}
\end{center}
\end{block}
}
\subsection{Feature Extraction}
\frame{
\frametitle{Feature Extraction}
}
\frame{
\frametitle{DBNF}
}
\subsection{DNN structure}
\frame{
\frametitle{DNN Structure}
}
\section{Demo}
\frame{
\frametitle{Demo}
}
\section{Post Processing}
\frame{
\frametitle{Evaluation Metrics}
}
\subsection{Smoothing Filters}
\frame{
\frametitle{Smoothing Filters}
}
\section{Results}
\frame{
\frametitle{Results}
}
\frame{
\frametitle{Future Work}
}
\frame{
\Huge{Questions?}
}


\frame[allowframebreaks]{
\frametitle{References}
\bibliography{example}
%\bibliographystyle{plain} %does not render "url" fields...
\bibliographystyle{IEEEtran}  %does render "url" fields, requires "_"s and "#"s to be escaped, e.g. "\_".
}
\frame{
\frametitle{Theory of LID}
\begin{itemize}
\item Given input \(S = \{ s(1), s(2), \dots , s(T)\}\), \(\{L_1,L_2,\dots,L_N\}\)
\end{itemize}
\pause
\begin{exampleblock}{Formulation of LID Task\cite{6451097}}
\(L^O = \argmax_I p(S|L_I)\)\newline
With segmentation into phones \(\upsilon\) (sequence \(\Upsilon\)):\newline
\pause
\(L^O = \argmax_I P(\Upsilon|L_I)\)\newline
With Viterbi decoding on set of phone models \(M\):\newline
\pause
\(\Upsilon^O = \argmax_{\upsilon} P(S|\upsilon, M)\)\newline
Combining last 2 equations:\newline
\pause
\cfbox{red}{
\begin{minipage}{4 in}
\(L^O = \argmax_{I}\sum_{\forall \Upsilon}P(S|\Upsilon, M)P(\Upsilon|L_I)\)
\end{minipage}}
\end{exampleblock}
}
\end{document}
